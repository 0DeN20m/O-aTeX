\documentclass{jlreq}

\usepackage{mystyle}
\usepackage{cleveref}
\usepackage{autonum}

\newcommand{\Term}[2][\relax]{%
	\textsf{#2} \if#1\relax\else{(\textsf{#1})}\fi%
}

\crefname{equation}{式}{式}
\crefname{proposition}{命題}{命題}
\crefname{theorem}{定理}{定理}
\crefname{lemma}{補題}{補題}
\crefname{lstlisting}{コード}{コード}
\crefname{figure}{図}{図}
\crefname{example}{例}{例}
\crefname{definition}{定義}{定義}

\newlist{defenum}{enumerate}{1}
\setlist[defenum]{label=\arabic*.\ ,ref=\thedefinition.\arabic*}
\crefalias{defenumi}{definition}

\newlist{propenum}{enumerate}{1}
\setlist[propenum]{label=\arabic*.\ ,ref=\theproposition.\arabic*}
\crefalias{propenumi}{proposition}

\newlist{thmenum}{enumerate}{1}
\setlist[thmenum]{label=\arabic*.\ ,ref=\thetheorem.\arabic*}
\crefalias{thmenumi}{theorem}

\newcommand{\crefpairconjunction}{と}
\newcommand{\crefrangeconjunction}{から}
\newcommand{\crefmiddleconjunction}{,}
\newcommand{\creflastconjunction}{,および}

\newcommand{\Inverse}[1]{{#1}^{-1}}
\newcommand{\LambdaMap}{\mathop{\lambda}\nolimits}

\begin{document}
	
	\setcounter{section}1
	\begin{definition}
		集合$A$とその上の二項演算$\cdot, \circ$が以下の条件を満たすとき,
		三つ組$\Tuple{A, \cdot, \circ}$を\Term{skew (left) brace}と呼ぶ:
		\begin{defenum}
			\item $\Tuple{A, \cdot}$と$\Tuple{A, \circ}$は群である.
			\item 任意の$a, b, c \in A$が\Term[brace relation]{brace関係式}
			\begin{equation}
				a \circ (b \cdot c) = (a \circ b) \cdot \Inverse{a} \cdot (a \circ c)
			\end{equation}
			を満たす.
			ここで,
			$\Inverse{a}$は$a$の$\cdot$に関する逆元である.
		\end{defenum}
	\end{definition}
	
	以下,
	$\Tuple{A, \cdot, \circ}$はskew braceとする.
	
	群$\Tuple{A, \cdot}$を\Term[additive group]{$A$の加法群}と呼び,
	群$\Tuple{A, \circ}$を\Term[multiplicative group]{$A$の乗法群}と呼ぶ.
	$a \in A$の乗法に関する逆元を$\overline{a}$で表す.
	
	加法$\cdot$と乗法$\circ$が混ざった式においては,
	加法を優先的に計算すると取り決めておく.
	また,
	加法の記号は適宜省略する.
	この取り決めによって,
	例えば,
	brace関係式は
	\begin{equation}
		a \circ bc = (a \circ b) \Inverse{a} (a \circ c)
	\end{equation}
	と表される.
	
	\begin{proposition}
		$\Tuple{A, \cdot}$の単位元と$\Tuple{A, \circ}$の単位元は一致する.
	\end{proposition}
	
	$A$の単位元を$1$で表す.
	
	$A$上の二項演算$\OpOperator{\cdot}$を
	\begin{equation}
		a \Op{\cdot} b = ba
	\end{equation}
	と定めると,
	三つ組$\Tuple{A, \Op{\cdot}, \circ}$はskew braceとなる.
	このskew braceを$\Op{A}$で表す.
	
	任意の$a \in A$に対して,
	写像$\LambdaMap_a \colon A \to A$を
	\begin{equation}
		\LambdaMap_a(b) = \Inverse{a} (a \circ b)
	\end{equation}
	で定める.
	さらに,
	$A$に関する\Term[lambda map]{ラムダ写像}$\LambdaMap \colon A \to \Maps{A}{A}$を
	\begin{equation}
		\LambdaMap(a) = \LambdaMap_a
	\end{equation}
	で定める.
	
	\begin{proposition}\label{prop:lambda_map}
		以下が成り立つ:
		\begin{propenum}
			\item ラムダ写像$\LambdaMap$は乗法に関する群準同型である.
			\item 任意の$a \in A$に対して,
			$\LambdaMap_a \in \Aut \Tuple{A, \cdot}$である.
		\end{propenum}
	\end{proposition}
	
	\begin{proof}
		\quad
		\begin{enumerate}
			\item 任意の$a, b, c \in A$に対して,
			\begin{align}
				\left(\LambdaMap_a \circ \LambdaMap_b\right)(c)
				&= \Inverse{a} (a \circ \Inverse{b} (b \circ c)) \\
				&= \Inverse{a} (a \circ \Inverse{b}) \Inverse{a} (a \circ b \circ c) \\
				&= \Inverse{(a \circ b)}((a \circ b) \circ c) \\
				&= \LambdaMap_{a \circ b}(c)
			\end{align}
			が成り立つ.
			したがって,
			$\LambdaMap_{a \circ b} = \LambdaMap_a \circ \LambdaMap_b$だから,
			$\LambdaMap$は群準同型写像である.
			\item 任意の$b, c \in A$に対して,
			\begin{align}
				\LambdaMap_a(bc)
				&= \Inverse{a} (a \circ bc) \\
				&= \Inverse{a} (a \circ b) \Inverse{a} (a \circ c) \\
				&= \LambdaMap_a(b) \LambdaMap_a(c)
			\end{align}
			が成り立つから,
			$\LambdaMap_a \in \End \Tuple{A, \cdot}$である.
			また,
			$\LambdaMap_{\overline{a}} = \Inverse{\LambdaMap_a}$なので,
			$\LambdaMap_a$は群同型写像である.\qedhere
		\end{enumerate}
	\end{proof}
	
	$\Op{A}$に関するラムダ写像を$\Op{\LambdaMap}$で表す.
	\cref{prop:lambda_map}より,
	$\Op{\LambdaMap}$は$\Tuple{A, \circ}$から$\Aut \Tuple{A, \Op{\cdot}} = \Aut \Tuple{A, \cdot}$への群準同型写像である.
	また,
	\begin{equation}
		\Op{\LambdaMap}_a (b) = (a \circ b) \Inverse{a}
	\end{equation}
	である.
	
	$\Tuple{\Tuple{A, \cdot}, \Op{\LambdaMap}}$は$\Tuple{A, \circ}$の群としての作用であるから,
	半直積$\Lambda_A = \Tuple{A, \cdot} \rtimes_{\Op{\LambdaMap}} \Tuple{A, \circ}$が定まる.
	
	\begin{definition}
		$X$を集合とする.
		$X$への2つの作用$\beta \colon \Tuple{A, \cdot} \to \Hom(X, X)$と$\rho \colon \Tuple{A, \circ} \to \Hom(X, X)$が,
		関係式
		\begin{equation}
			\beta(\Op{\LambdaMap}_a(b)) = \rho(a) \circ \beta(b) \circ \Inverse{\rho(a)}
		\end{equation}
		を満たすとき,
		三つ組$\Tuple{X, \beta, \rho}$を$A$の作用と呼ぶ.
	\end{definition}
	
	$\Lambda_A$は自然に$A$集合となる.
	
	$X$を$A$集合とする.
	
	\begin{proposition}
		$X$を$A$集合としたとき,
		以下が成り立つ:
		\begin{propenum}
			\item 任意の$a \in A$と$x \in X$に対して,
			$a \circ A \cdot x = A \cdot (a \circ x)$が成り立つ.
			\item $A \cdot (A \circ x)$は$A$集合である.
			\item 
		\end{propenum}
	\end{proposition}
	
\end{document}