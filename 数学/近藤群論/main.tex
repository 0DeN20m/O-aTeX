\documentclass{jlreq}
\usepackage{my-common}

% 本の体裁に合わせる

\RenewBlockHeading{section}{1}{%
  font=\bfseries\Large,%
  label_format=第\thesection 章%
}
\RenewBlockHeading{subsection}{2}{%
  font=\bfseries\large,%
  label_format=\S\thesubsection%
}
\renewcommand{\thesubsubsection}{\alph{subsubsection}}
\RenewBlockHeading{subsubsection}{3}{%
  font=\bfseries\rmfamily,%
  label_format=\thesubsubsection)%
}
\SetBlockHeadingSpaces{
  [lines=5]{_section,23pt,_subsection,16pt,_subsubsection,8pt}%
}

\begin{document}

\section{群論}

\subsection{群の概念}

\subsubsection{半群}

$V = \tuple{V, \innerprod{\dummy}{\dummy}}$を内積空間とする.
このとき,$v \in V$上のノルム$\norm{v}$を
\begin{equation}
  \norm{v} = \sqrt{\innerprod{v}{v}}
\end{equation}
で定義すれば,$V$上の距離$\distance{d}$が
\begin{equation}
  \distance{d}(u, v) = \norm{v - u}
\end{equation}
で定められる.このとき,等長変換群$\Isom V$は
\begin{equation}
  \Isom V =
\end{equation}
で与えられる.

\end{document}