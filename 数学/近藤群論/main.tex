\documentclass{jlreq}
\usepackage{my-common}

% 本の体裁に合わせる

\RenewBlockHeading{section}{1}{%
  font=\bfseries\Large,%
  label_format=第\thesection 章%
}
\RenewBlockHeading{subsection}{2}{%
  font=\bfseries\large,%
  label_format=\S\thesubsection%
}
\renewcommand{\thesubsubsection}{\alph{subsubsection}}
\RenewBlockHeading{subsubsection}{3}{%
  font=\bfseries\rmfamily,%
  label_format=\thesubsubsection)%
}
\SetBlockHeadingSpaces{
  [lines=5]{_section,23pt,_subsection,16pt,_subsubsection,8pt}%
}

\makeindex

\begin{document}

\section{群論}

\subsection{群の概念}

\subsubsection{半群}

\begin{definition}
  $S$を集合とする.
  写像$S \times S \to S$が定まっているとき,$S$の上に\term[えんざんがていぎされている]{演算が定義されている}という.
\end{definition}

$S$の上に演算$f$が定義されているとき,任意の$x, y \in S$に対して,
\begin{equation}
  xy = f(x, y)
\end{equation}
と表すことが多い.

\subsection{剰余類と剰余群}

\subsection{準同型写像}

\subsubsection{準同型と同型}

$V = \tuple{V, \innerproduct{\dummy}{\dummy}}$を内積空間とする.
このとき,$v \in V$上のノルム$\norm{v}$を
\begin{equation}
  \norm{v} = \sqrt{\innerproduct{v}{v}}
\end{equation}
で定義すれば,$V$上の距離$\distance{d}$が
\begin{equation}
  \distance{d}(u, v) = \norm{v - u}
\end{equation}
で定められる.このとき,
\begin{equation}
  \isometries V = \{ f \colon \}
\end{equation}
を\term[とうちょうへんかんぐん]{等長変換群}[isometory]と呼ぶ.

\printindex

\end{document}