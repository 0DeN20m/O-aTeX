\documentclass{jlreq}
\usepackage{common}

\jlreqsetup{}

\begin{document}

\Term[すうがく]{数学}[mathematics, math]

\part{見出しの確認}[副題あり]

パートの文章.

\section{セクション}[副題]

セクションの文章.

\subsection{サブセクション}[副題]

サブセクションの文章.

\subsubsection{サブサブセクション}[副題]

サブサブセクションの文章.

\part{見出しの確認}

パートの文章.

\section{セクション}

セクションの文章.

\subsection{サブセクション}

サブセクションの文章.

\subsubsection{サブサブセクション}

サブサブセクションの文章.

\part{本文の確認}

\section{定理環境}

\begin{axiom}[コメント]
  公理の内容.
\end{axiom}

\begin{axiom*}
  公理の内容.
\end{axiom*}

\begin{definition}[コメント]
  定義の内容.
\end{definition}

\begin{definition*}
  定義の内容.
\end{definition*}

\begin{theorem}[コメント]
  定理の内容.
\end{theorem}

\begin{theorem*}
  定理の内容.
\end{theorem*}

\begin{proposition}[コメント]
  命題の内容.
\end{proposition}

\begin{proposition*}
  命題の内容.
\end{proposition*}

\begin{lemma}[コメント]
  補題の内容.
\end{lemma}

\begin{lemma*}
  補題の内容.
\end{lemma*}

\begin{corollary}[コメント]
  系の内容.
\end{corollary}

\begin{corollary*}
  系の内容.
\end{corollary*}

\begin{example}[コメント]
  例の内容.
\end{example}

\begin{example*}
  例の内容.
\end{example*}

\part{サンプル}

GitHub Copilot によって生成されたサンプル文章.

\section{群}

\subsection{定義}

群は,様々な数学的対象の対称性を記述するための基本的な構造である.

\begin{definition}
  群とは,集合 $G$ と二項演算 $\cdot : G \times G \to G$ の組 $(G, \cdot)$ であって,次の公理を満たすものをいう:
  \begin{enumerate}
    \item (結合律) 任意の $a, b, c \in G$ に対して,$(a \cdot b) \cdot c = a \cdot (b \cdot c)$ が成り立つ.
    \item (単位元の存在) $G$ に属する元 $e$ が存在して,任意の $a \in G$ に対して,$e \cdot a = a \cdot e = a$ が成り立つ.
    \item (逆元の存在) 任意の $a \in G$ に対して,$G$ に属する元 $b$ が存在して,$a \cdot b = b \cdot a = e$ が成り立つ.
  \end{enumerate}
\end{definition}

\begin{example}
  整数全体の集合 $\mathbb{Z}$ と加法 $+$ は群 $(\mathbb{Z}, +)$ を形成する.
\end{example}

\begin{example}
  非零実数全体の集合 $\mathbb{R}^*$ と乗法 $\times$ は群 $(\mathbb{R}^*, \times)$ を形成する.
\end{example}

\subsection{諸性質}

群の基本的な性質についていくつか述べる.

\begin{theorem}
  群において,単位元は一意である.
\end{theorem}

\begin{proof}
  群 $(G, \cdot)$ において,単位元が $e$ と $e'$ の二つ存在すると仮定する.すると,任意の $a \in G$ に対して,$e \cdot a = a$ および $e' \cdot a = a$ が成り立つ.特に,$a = e'$ とすると,$e \cdot e' = e'$ となる.同様に,$a = e$ とすると,$e' \cdot e = e$ となる.したがって,$e = e'$ が成り立ち,単位元は一意であることが示された.
\end{proof}

\end{document}