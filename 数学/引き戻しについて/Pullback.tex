\documentclass{jlreq}

\usepackage{mystyle,tikz-cd}

\title{引き戻しについて}
\author{著者}
\date{\today}

\begin{document}

	\maketitle

	\begin{abstract}
		引き戻しの圏論的な基礎です.
	\end{abstract}

	\section{引き戻しの定義}
	
	集合$X, Y, Z$と写像$f \colon X \rightarrow Z, g \colon Y \rightarrow Z$に対して,集合
	\begin{equation*}
		X {}_f\times_g Y = \{\, (x,y) \in X \times Y \mid f(x) = g(y) \,\}
	\end{equation*}
	と自然な射影$\pi_f \colon X {}_f\times_g Y \rightarrow X,\ \pi_g \colon X {}_f\times_g Y \rightarrow Y$の組$(X {}_f\times_g Y, \pi_f, \pi_g)$の組を,$f,g$の引き戻しという.
	このとき,図式
	\begin{equation*}
		\begin{tikzcd}
			X {}_f\times_g Y \arrow[r, "\pi_g"] \arrow[d, "\pi_f"'] & Y \arrow[d, "g"] \\
			X \arrow[r, "f"']                                       & Z               
		\end{tikzcd}
	\end{equation*}
	が可換になる.
	
	引き戻しは,この可換図式によって一般の圏に対して定義される.
	
	\begin{definition}
		$\mathcal{C}$を圏とする.
		2つの射$f \in \Hom_{\mathcal{C}}(X,Z),\ g \in \Hom_{\mathcal{C}}(Y,Z)$に対して,極限$(L=\Lim(\begin{tikzcd}[cramped, sep=scriptsize]
			X \arrow[r, "f"] & Z & Y \arrow[l, "g"']
		\end{tikzcd}), \pi_f, \pi_g)$を$f,g$の$\mathcal{C}$における引き戻しという.
		つまり,図式
		\begin{equation*}
			\begin{tikzcd}
				L \arrow[r, "\pi_g"] \arrow[d, "\pi_f"'] & Y \arrow[d, "g"] \\
				X \arrow[r, "f"']                                       & Z               
			\end{tikzcd}
		\end{equation*}
		を可換にし,特に普遍性を持つものを$f,g$の引き戻しという.
	\end{definition}
	
	組$(P, p, q)$が$f,g$の引き戻しであることを,可換図式において
	\begin{equation*}
		\begin{tikzcd}
			P \arrow[r, "q"] \arrow[d, "p"'] \arrow[rd, "\mathrm{PB}", phantom] & Y \arrow[d, "g"] \\
			X \arrow[r, "f"']                                                   & Z               
		\end{tikzcd}, \qquad
		\begin{tikzcd}
			P \arrow[r, "q"] \arrow[d, "p"'] \arrow[rd, "\square", phantom] & Y \arrow[d, "g"] \\
			X \arrow[r, "f"']                                               & Z               
		\end{tikzcd}
	\end{equation*}
	などと表す.
	
	圏$\mathcal{C}$に直積とイコライザーが存在すれば,引き戻しが構成できる.
	ここで,$f,g \in \Hom_{\mathcal{C}}(X,Y)$のイコライザー$(E, \pi \colon E \rightarrow X)$とは,$f \pi = g \pi$を満たすもののうち,普遍性を持つもののことをいう.
	
	\begin{proposition}
		$\mathcal{C}$を圏とする.
		$f \in \Hom_{\mathcal{C}}(X,Z),\ g \in \Hom_{\mathcal{C}}(Y,Z)$に対して,$X,Y$の直積$(X \times Y, p_X, p_Y)$と$\begin{tikzcd}[cramped, sep=scriptsize]
			X \arrow[r, "fp_X", shift left] \arrow[r, "gp_Y"', shift right] & Z
		\end{tikzcd}$のイコライザー$(E, \pi)$が存在するとする.
		このとき,$(E, p_X \pi, p_Y \pi)$は$f,g$の引き戻しである.
	\end{proposition}
	
	\begin{proof}
		イコライザーの定義より$f (\pi p_X) = g (\pi p_Y)$が成り立つので,図式
		\begin{equation*}
			\begin{tikzcd}
				E \arrow[r, "p_Y \pi"] \arrow[d, "p_X \pi"'] & Y \arrow[d, "g"] \\
				X \arrow[r, "f"']                            & Z               
			\end{tikzcd}
		\end{equation*}
		は可換になる.
		よって,$(E, p_X \pi, p_Y \pi)$が普遍性を持つことを示せばよい.
		
		任意に対象$D \in \mathcal{C}$と射$\rho_X \in \Hom_{\mathcal{C}}(D, X),\ \rho_Y \in \Hom_{\mathcal{C}}(D, Y)$を取り,$f \rho_X = g \rho_Y$だと仮定する.
		このとき,直積の普遍性から,ただ一つの$p \in \Hom_{\mathcal{C}}(D, X \times Y)$が存在して,$\rho_X = p_X p,\ \rho_Y = p_Y p$が成り立つ.
		さらに,$(f p_X) p = f \rho_X = g \rho_Y = (g p_Y) p$が成り立つから,イコライザーの普遍性より,ただ一つの$e \in \Hom_{\mathcal{C}}(D, E)$が存在して,$p = \pi e$が成り立つ.
		以上より,可換図式
		\begin{equation*}
			\begin{tikzcd}
				D \arrow[rrddd, "\rho_X"', bend right] \arrow[rrrdd, "\rho_Y", bend left] \arrow[rrdd, "p" description, dashed, bend right] \arrow[rd, "e", dashed] &                     &                                               &                  \\
				& E \arrow[rd, "\pi"] &                                               &                  \\
				&                     & X \times Y \arrow[d, "p_X"'] \arrow[r, "p_Y"] & Y \arrow[d, "g"] \\
				&                     & X \arrow[r, "f"']                             & Z               
			\end{tikzcd}
		\end{equation*}
		を得る.
		したがって,可換図式
		\begin{equation*}
			\begin{tikzcd}
				D \arrow[rdd, "\rho_X"', bend right] \arrow[rrd, "\rho_Y", bend left] \arrow[rd, "e" description, dashed] &                                              &                  \\
				& E \arrow[r, "p_Y \pi"] \arrow[d, "p_X \pi"'] & Y \arrow[d, "g"] \\
				& X \arrow[r, "f"']                            & Z               
			\end{tikzcd}
		\end{equation*}
		を得るから,$(E, \pi p_X, \pi p_Y)$は普遍性を持つ.
	\end{proof}
	
	一般論から,引き戻しは存在すれば同型を除いて一意に存在する.
	
	\section{引き戻しの例}
	
	\begin{example}
		$\mathcal{C}$を圏,$T \in \mathcal{C}$を終対象とする.
		任意の対象$C \in \mathcal{C}$に対して,ただ一つ存在する$\mathcal{C}$の射$C \rightarrow T$を$!_C$で表す.
		このとき,任意の対象$X, Y \in \mathcal{C}$に対して,$!_X, !_Y$の引き戻しは,$X, Y$の直積となる.
		
		実際,$(X \times Y, \id_{X \times Y})$は$\begin{tikzcd}[cramped, sep=scriptsize]
			X \times Y \arrow[r, "!_{X \times Y}", shift left] \arrow[r, "!_{X \times Y}"', shift right] & T
		\end{tikzcd}$のイコライザーを与えている.
		よって,$(X \times Y, p_X, p_Y)$は引き戻しである.
	\end{example}
	
	\begin{example}
		集合の圏$\mathbf{Set}$における引き戻しは,冒頭に述べた$(X {}_f\times_g Y, \pi_F, \pi_g)$である.
	\end{example}
	
	\begin{example}
		$G$を有限群とする.
		有限$G$集合の圏において,図式
		$\begin{tikzcd}[cramped, sep=scriptsize]
			G/H \arrow[r, "f"] & G/L & G/K \arrow[l, "g"']
		\end{tikzcd}$
		の引き戻しは
		\begin{equation*}
			\bigsqcup_{HxK \in H \backslash aLb^{-1} / K} G/(H \cap {}^xK), \quad \pi_f \colon yH \cap {}^xK \mapsto yH, \quad \pi_g \colon yH \cap {}^xK \mapsto y x^{-1} K
		\end{equation*}
		で与えられる.
		ただし,$f( H ) = aL,\ g( K ) = bL$である.
	\end{example}
	
	\bibliographystyle{jplain}
	\bibliography{myrefs-jp}
	\nocite{Nakaoka:圏論の技法}

\end{document}
