\documentclass{jlreq}

\usepackage{mystyle,bm}

\title{特性方程式について}
\author{著者}
\date{\today}

\begin{document}

	\maketitle

	\begin{abstract}
		漸化式の特性方程式について書きました.
	\end{abstract}

	\section{特性方程式について}
	
	$N \ge 1$に対して,$N+1$項間斉次漸化式
	\begin{equation}	\label{eq:recurrence_formula}
		a_{n+N} = \sum_{k=0}^{N-1} p_{k} a_{n+k} \quad (n \ge 0)
	\end{equation}
	を考える.
	\begin{equation*}
		\bm{a}_n = \begin{pmatrix}
			a_{n+N-1} \\
			a_{n+N-2} \\
			\vdots \\
			a_n
		\end{pmatrix}, \quad A = \begin{pmatrix}
		p_{N-1} & \ldots & p_1 & p_0 \\
		1 &  &  & 0 \\
		& \ddots &  & \vdots \\
		&  & 1 & 0
		\end{pmatrix}
	\end{equation*}
	とおけば,漸化式\eqref{eq:recurrence_formula}は
	\begin{equation*}
		\bm{a}_{n+1} = A \bm{a}_n \quad (n \ge 0)
	\end{equation*}
	と表される.
	したがって,
	\begin{equation*}
		\bm{a}_n = A^n \bm{a}_0
	\end{equation*}
	が成り立つから,$A^n$が分かれば$a_n$も分かることになる.
	
	行列の冪を求める方法の一つに固有多項式の利用がある.
	$A$の固有多項式は
	\begin{equation*}
		t^N - \sum_{k=0}^{N-1} p_k t^k
	\end{equation*}
	であり,あたかも\eqref{eq:recurrence_formula}の数列部分を$t$に置き換えたような形をしている.
	これを漸化式\eqref{eq:recurrence_formula}の特性多項式という.
	特性多項式を$\Phi(t)$としたとき,方程式$\Phi(t) = 0$を特性方程式という.
	
	\section{2項間漸化式の固有多項式}
	
	\eqref{eq:recurrence_formula}において,$N=1$とした場合を考える.
	このときの特性方程式は
	\begin{equation*}
		t-p_0 = 0
	\end{equation*}
	である.
	
	対して,定数項として$q$を加えた漸化式
	\begin{equation*}
		a_{n+1} = p_0 a_n + q
	\end{equation*}
	について,その特性方程式が
	\begin{equation}\label{eq:fake_char_poly}
		t = p_0 t + q
	\end{equation}
	であると学んだ人も多いと思う.
	すでに見た通り,特性方程式とは,そもそも斉次線形漸化式に対して定義されるものである.
	したがって,方程式\eqref{eq:fake_char_poly}を特性多項式と呼ぶのは適切ではない.
\end{document}
